\documentclass[12pt,a4paper]{article}

\usepackage[brazil]{babel}
\usepackage[utf8]{inputenc}
\usepackage{enumerate}
\usepackage{float}
\usepackage{amssymb}
\usepackage{amsthm}
\usepackage{amsmath}
\usepackage{minted}
\usepackage{color}

\newcommand{\norm}[1]{\left\lVert#1\right\rVert}

\begin{document}
\begin{center}
  {\LARGE Problemas de geometria}
\end{center}
\bigskip 

Implemente as funções declaradas no arquivo \texttt{geometria.h} (veja a listagem do código desse arquivo abaixo).

\inputminted{C}{geometria.h}

\section* {Observações}

\textcolor{red}{\textbf{Atenção:} Siga as instruções no arquivo
README para compilação. Em particular, seu código deverá incluir
\texttt{geometria.h} via a linha
\begin{verbatim}
#include "geometria.h"
\end{verbatim}
no início do programa. É proibido modificar o conteúdo desse arquivo de cabeçalhos. 

As funções foram declaradas numa certa ordem. A ordem foi pensada de
modo que, na hora de implementar uma função é bem provável que você
precise chamar alguma função que foi declarada antes dela (exceto
pelas primeiras que são muito simples).

\subsection* {Geometria analítica}
Se você não se lembra de alguma definição, pode pesquisar no Google,
ou no livro que você usou para o curso de geometria analítica.
Algumas lembranças sobre conceitos de geometria analítica serão dadas
em sala, durante a aula prática. Algumas funções da lista só podem ser
implementadas de um jeito, outras têm soluções variadas, vai depender
da modelagem matemática que você adotar para resolver cada problema
geométrico associado.

% Se você não se lembra de geometria analítica, aqui vão algumas dicas.
% \begin{itemize}
% \item As definições que você esqueceu você pode consultar no Google.
% \item O cosseno do ângulo $\Theta$ entre dois vetores pode ser
%   calculado através da lei dos cossenos
%   \[
%   c^2 = a^2 + b^2 - 2ab\,\mathrm{cos}(\Theta)
%   \]
% onde c é o comprimento lado oposto ao ângulo $\Theta$ no triângulo formado pelos dois vetores, e $a$ e $b$ são os comprimentos dos vetores. Mas primeiro expanda e depois simplifique a expressão que você encontrar para o cose-no, pois o que queremos saber do cose-no é só o sinal! A expressão final depois de simplificada fica \emph{bem} simples!
% \item Nas funções \texttt{cruza} e \texttt{dentro}, você deve usar apenas algumas chamadas à função \texttt{sentido} e uns poucos operadores lógicos.
% \item Para a função \texttt{projeta}, lembre-se de que, se projetamos um vetor $u$ sobre um vetor $v$, o vetor resultante é um vetor múltiplo de $v$, digamos $\alpha v$. Portanto, a dica é: você deve encontrar $\alpha$ que minimiza
%   \[
%   f(\alpha) = \norm{u - \alpha v}.
%   \]
%   Outra dica é que, para minimizar a raiz quadrada de algo que é sempre não negativo, basta minimizar esse algo!
 
% \end{itemize}

\end{document}
